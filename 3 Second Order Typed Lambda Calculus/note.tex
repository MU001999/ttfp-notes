\documentclass[UTF8]{article}
\usepackage{ctex}
\usepackage{amssymb}
\usepackage{amsmath}
\usepackage{graphicx}
\newtheorem{thm}{定义}[section]
\newtheorem{notation}[thm]{记号}
\newtheorem{lemma}[thm]{引理}

\title{二阶(second order)类型化的$\lambda$-演算\\[2ex]\begin{large}读书笔记\end{large}}
\author{许博}
\date{}

\begin{document}
\maketitle
	\section{疑惑}
	
	1. 所谓依赖是否只存在于抽象,如$\lambda x:\sigma.M$依赖于$x$,而$MN$并不依赖于$M$或$N$?

	\section{类型抽象和类型应用}
		\noindent
		在$\lambda{\rightarrow}$中,遇到的抽象和应用都是在项层面,也即所谓的一阶:
		
		\noindent
		- 如在抽象过程中,项$M$上$x$的抽象,其中假设$x:\sigma$,$\lambda x:\sigma.M$依赖于项$x$。
		
		因此,在$\lambda{\rightarrow}$中,可以构建依赖于项的项(terms depending on terms)。
			
		\noindent
		- 对应于抽象的是应用,项$MN$是应用项$M$到项$N$的结果。
		
		$\lambda{\rightarrow}$中的抽象是一阶抽象,或一阶依赖,因为抽象是在项上进行的。应用也是一阶的。
		
		本章将引入依赖于类型的项(terms depending on types),相关的运算称之为二阶运算或二阶依赖。而所获得的系统称为二阶类型化的$\lambda$-演算,记为$\lambda{2}$。
		
		从一个例子,来看$\lambda{\rightarrow}$表达能力带来的限制,比如对于接收一个输入然后直接返回输入自身的恒等函数:
		
		\noindent
		- 对于自然数,nat,恒等函数为$\lambda x:nat.x$。
		
		\noindent
		- 对于布尔值,bool,它是$\lambda x:bool.x$。
		
		\noindent ...
		
		对于每一个类型,都有一个与之对应的恒等函数,而在$\lambda{\rightarrow}$中,通用的恒等函数无法表示,尽管对于所有类型,它的行为是相同的。为了表示通用的(恒等)函数,加入另一种抽象:
		
		$\lambda\alpha:*.\lambda x:\alpha.x$。
		
		新加入的抽象是出现在第一个$\lambda$之后的类型变量$\alpha$,而符号*表示所有(简单)类型的类型,换而言之,如果$\alpha$是一个类型,那么$\alpha\in *$。需要注意的是,$\lambda\alpha:*.\lambda x:\alpha.x$是一个依赖于类型的项(a  term depending on a type),它所依赖的类型是$\alpha$。
		
		得到的这个(二阶)项称为多态的(polymorphic)恒等函数,需要注意的是它自己并非一个恒等函数,而是一个潜在的(potential)恒等函数。在进行(二阶)应用以及$\beta$-规约后可以得到一个名副其实(genuine)的恒等函数,如:
		
		\noindent
		- $(\lambda\alpha:*.\lambda x:\alpha.x)nat\rightarrow_\beta\lambda x:nat.x$,得到一个在$nat$上的恒等函数。
		
		\noindent
		- $(\lambda\alpha:*.\lambda x:\alpha.x)(nat\rightarrow bool)\rightarrow_\beta\lambda x:(nat\rightarrow bool).x$,得到一个在$nat\rightarrow bool$上的恒等函数。
		
		回顾类型变量的无限集合:$\mathbb{V} = \{\alpha, \beta, \gamma, ...\}$。
		
		\begin{thm} 所有简单类型的集合$\mathbb{T}$
			
			(1)(类型变量)如果$\alpha \in \mathbb{V}$,则 $\alpha \in \mathbb{V}$。
			
			(2)(箭头类型)如果$\sigma,\tau\in\mathbb{T}$,则$(\sigma\rightarrow\tau)\in\mathbb{T}$。
			
		\end{thm}
	
		上一个例子中,$\alpha$不能以$\sigma$替换,因为$\sigma$表示一个尽管未知但具体唯一确定的类型,作为某一具体类型的符号表示,而$\alpha$是一个类型变量,可被未知的任意类型替换,但在替换之前,其本身是一个类型变量,而非一个具体的类型。
		
		因此当以这种方式扩展$\lambda{\rightarrow}$时,需要引入二阶抽象和应用,除此之外还需要对于二阶项的$\beta$-规约。
		
		第二个是关于迭代(iteration)的例子,也即一个函数的重复应用。对于一个类型$\sigma$和一个具有类型$\sigma\rightarrow\sigma$的函数$F$,定义$D_{\sigma,F}$为映射类型$\sigma$的$x$到$F(F(x))$的函数。$D_{\sigma,F}$是$F$的二次迭代(second iteration),也可记为$F\circ F$(F和自己的复合函数)。
		
		在$\lambda{\rightarrow}$中$D_{\sigma,F}$可以通过项$\lambda x:\sigma.F(F x)$表示,而若要定义为任意的$\sigma$以及任意的$F:\sigma\rightarrow\sigma$定义通用的$D$,则需要使用类型变量$\alpha$而不是固定的类型$\sigma$,使用满足$f:\alpha\rightarrow\alpha$项变量$f$而不是固定的函数$F$。通过由$f$和$\alpha$的抽象可以得到$D$的定义:
		
		$D\equiv\lambda\alpha:*.\lambda f:\alpha\rightarrow\alpha.\lambda x:\alpha.f(fx)$。
		
		同样可以给出函数符合运算$\circ$在$\lambda{2}$中的定义:
		
		$\circ\equiv\lambda\alpha:*.\lambda\beta:*.\lambda\gamma:*.\lambda f:\alpha\rightarrow\beta.\lambda g:\beta\rightarrow\gamma.\lambda x:\alpha.g(fx)$。
		
	\section{$\Pi$-类型}
		$\lambda{2}$中的二阶项,如$\lambda\alpha:*.\lambda x:\alpha.x$,同样具有类型,表示为$\Pi\alpha:*.\alpha\rightarrow\alpha$,也即$\lambda\alpha:*.\lambda x:\alpha.x\ :\ \Pi\alpha:*.\alpha\rightarrow\alpha$。其中$\Pi\alpha:*$表示$\alpha$为绑定(类型)变量,且它的类型是$*$。
		
		在引入$\Pi$-类型之前,等同项$\lambda\alpha:*.\lambda x:\alpha.x$和$\lambda\beta:*.\lambda x:\beta.x$的类型因为没有标识绑定(类型)变量而不同。
	
	\section{二阶抽象和应用规则}
		由于引入了二阶抽象和二阶应用以及$\Pi$-类型,因此对于$\lambda{\rightarrow}$的推导系统也要进行相应的扩展。
		
		\begin{thm}(二阶抽象规则)\\
			
			${\rm(}abst_2{\rm)}$$\ \dfrac{\Gamma,\alpha:*\vdash M:A}{\Gamma\vdash\lambda\alpha:*.M:\Pi\alpha:*.A}$
		\end{thm}
	
		\begin{thm}(二阶应用规则)\\
			
			${\rm(}appl_2{\rm)}$$\ \dfrac{\Gamma\vdash M:\Pi\alpha:*.A\ \ \ \Gamma\vdash B:*}{\Gamma\vdash MB:A[\alpha:=B]}$
		\end{thm}
	
		在二阶应用规则中,$B:*$意为$B$是一个类型。
		
	\section{系统$\lambda{2}$}
		首先扩展类型的定义,其中$\mathbb{V}$是类型变量的集合$\{\alpha,\beta,\gamma,...\}$:
		
		$\mathbb{T}2 = \mathbb{V}|(\mathbb{T}2\rightarrow\mathbb{T}2)|(\Pi\mathbb{V}:*.\mathbb{T}2)$。
		
		第二部,扩展预先类型化的$\lambda$-项的集合($\Lambda_{\mathbb{T}}$)以允许二阶抽象和应用:
		
		\begin{thm}(二阶预先类型化的$\lambda$-项,$\lambda{2}$-项,$\Lambda_{\mathbb{T}2}$)
			
			$\Lambda_{\mathbb{T}2} = V|(\Lambda_{\mathbb{T}2}\Lambda_{\mathbb{T}2})|(\Lambda_{\mathbb{T}2}\mathbb{T}2)|(\lambda V:\mathbb{T}2.\Lambda_{\mathbb{T}2})|(\lambda\mathbb{V}:*.\Lambda_{\mathbb{T}2})$。
		\end{thm}
	
		需要注意的是,其中有两类变量:对象(object)变量$V$和类型变量$\mathbb{V}$。由对象变量进行一阶抽象,由类型变量进行二阶抽象。同样有与之对应的一阶应用和二阶应用。
		
		\begin{thm}(语句(statement),声明)
			
			(1) 一个语句形如$M:\sigma$,其中$M\in\Lambda_{\mathbb{T}2}$且$\sigma\in\mathbb{T}2$,或形如$\sigma:*$,其中$\sigma\in\mathbb{T}2$。
			
			(2) 一个声明是由一个项变量或一个类型变量作为对象的语句。
		\end{thm}
		
		因为$\lambda{\rightarrow}$中的类型常量在$\lambda{2}$中变成了类型变量,所以与$\lambda{\rightarrow}$相比,$\lambda{2}$更加严格一些,因为所有的变量使用前必须被声明,在使用变量之前,我们知道所有变量的类型。
		
		\begin{thm}($\lambda{2}$-上下文;域;dom)
			
			(1) $\emptyset$是一个$\lambda{2}$-上下文;
			
			$dom(\emptyset) = ()$,空的列表。
			
			(2) 如果$\Gamma$是一个$\lambda{2}$-上下文,$\alpha\in\mathbb{V}$且$\alpha\notin dom(\Gamma)$,则$\Gamma,\alpha:*$是一个$\lambda{2}$-上下文;
			
			$dom(\Gamma,\alpha:*)=(dom(\Gamma),\alpha)$。
			
			(3) 如果$\Gamma$是一个$\lambda{2}$-上下文,如果$\rho\in\mathbb{T}2$且对于$\rho$中出现的所有自由类型变量$\alpha$都有$\alpha\in dom(\Gamma)$,以及如果$x\notin dom(\Gamma)$,则$\Gamma,x:\rho$是一个$\lambda{2}$-上下文;
			
			$dom(\Gamma,x:\rho)=(dom(\Gamma),x)$。
		\end{thm}
	
		需要注意的是,定义需要在一个$\lambda{2}$-上下文中的所有的项变量以及类型变量彼此之间都是不同的(符号上)。
		
		为了符合新的上下文定义,修改$\lambda{\rightarrow}$的($var$)-规则,以能够开始关联一个正确的$\lambda{2}$-上下文的变量的类型的推导:
		
		\begin{thm} ($\lambda{2}$的(变量,var)规则)
			
			(变量,var)如果$\Gamma$是一个$\lambda{2}$-上下文且$x:\sigma\in\Gamma$,则$\Gamma\vdash x:\sigma$。
		\end{thm}
	
		$\lambda{\rightarrow}$中的(appl)规则以及(abst)规则将继续使用而不进行修改。
		
		需要注意的是,上一节中我们没有机会使用定义的($appl_2$)-规则,因为其中的第二个$\bf premiss$是$\Gamma\vdash B:*$,但是没有规则可以产生这个形式的$\bf conclusion$,为了解决这个问题,需要添加一条规则:
		
		\begin{thm}(形成规则,formation rule)
			(form) 如果$\Gamma$是一个$\lambda{2}$-上下文,$B\in\mathbb{T}2$且$B$中所有的自由类型变量已经在$\Gamma$中声明,则$\Gamma\vdash B:*$。
		\end{thm}
		
		这个规则告诉了我们一个正确形式化的$\lambda{2}$-类型$B$的类型是什么。
		
		需要注意的是,($form$)-规则有三个副条件,但是没有$\bf premisses$,因此,它可以像($var$)-规则一样出现在推导树的叶子。
		
		\begin{thm}(合法的$\lambda{2}$-项)如果存在一个$\lambda{2}$-上下文$\Gamma$和一个类型$\rho\in\mathbb{T}2$且$\Gamma\vdash M:\rho$,则项$M\in\Lambda_{\mathbb{T}2}$称为合法的。
		\end{thm}
	
	\section{$\lambda{2}$的性质}
		调整$\alpha$-变换的定义,以适应$\Pi$-类型:
		
		\begin{thm}($\alpha$-变换或$\alpha$-等价,扩展的)
			
			(1a)(项变量的重命名)
			
			如果$y\notin FV(M)$且$y$不作为绑定变量在$M$中出现,则$\lambda x:\sigma.M=_\alpha\lambda y:\sigma.M^{x\rightarrow y}$。
			
			(1b)(类型变量的重命名)
			
			如果$\beta$不在$M$中出现,则$\lambda\alpha:*.M=_\alpha\lambda\beta:*.M\left[\alpha:=\beta\right]$。
			
			如果$\beta$不在$M$中出现,则$\Pi\alpha:*.M=_\alpha\Pi\beta:*.M\left[\alpha:=\beta\right]$。
			
			(2),(3a),(3b),(3c)(相容性,自反性,对称性,传递性)不变。
			
		\end{thm}
	
		扩展$\beta$-规约以满足$\lambda{2}$:
		
		\begin{thm}(一步$\beta$-规约,对于$\Lambda_2$-项的$\rightarrow_\beta$)
			
			(1a)(基础,Basis,一阶)$(\lambda x:\sigma.M)N \rightarrow_\beta M\left[x:=N\right]$
			
			(1b)(基础,Basis,二阶)$(\lambda\alpha:*.M)T\rightarrow_\beta M\left[\alpha:= T\right]$
			
			(2)(相容性)不变。
		\end{thm}
		
		需要注意的是排列定理(Permutation Lemma)不再允许对于声明的任意排列,因为类型变量的声明可能会依赖更早的声明,但如果排列后依然满足$\lambda{2}$-上下文,则性质依然保持。
		
\end{document}
