\documentclass[UTF8]{article}
\usepackage{ctex}
\usepackage{ulem}
\usepackage{amssymb}
\usepackage{amsmath}
\usepackage{graphicx}
\newtheorem{thm}{定义}[section]
\newtheorem{notation}[thm]{记号}
\newtheorem{lemma}[thm]{引理}

\title{依赖于类型的类型\\[2ex]\begin{large}读书笔记\end{large}}
\author{许博}
\date{}

\begin{document}
\maketitle
		\noindent
		本章将介绍另一个扩充$\lambda{\rightarrow}$的系统$\lambda{\uline{\omega}}$,而非在$\lambda{2}$上继续扩充的系统。
		
	\section{疑问}
		在4.7变换规则中:\\
		
		(conv) 如果$B=_\beta B'$,则$\dfrac{\Gamma\vdash A:B\ \ \ \Gamma\vdash B':s}{\Gamma\vdash A:B'}$。\\
		
		其中$B':s$,但是$\lambda{\uline{\omega}}$中并没有对类(kind)的抽象,如$\lambda\kappa:\square.M$,也没有相对应的应用,换句话说,当$s\equiv\square$时,所有的$B$与$B'$应当是恒等的(不包括$\beta$-等价),这样理解是否正确?

	\section{类型构造子}
		\noindent
		上一章中引入了一般性的(generalised)项,通过从类型变量抽象项。比如作用于确定类型$\sigma$的恒等函数$\lambda x:\sigma.x$可以被泛化成项$\lambda\alpha:*.\lambda x:\alpha.x$,多态的(polymorphic)恒等函数,抽象于$\alpha$。
		
		通过类似的方式也可以构造一般性的类型。比如形如$\beta\rightarrow\beta,\gamma\rightarrow\gamma,...$,等具有结构$\lozenge\rightarrow\lozenge$的类型,其中箭头的左边和右边是一样的类型。
		
		为了处理这种情况,引入一个包含了这种结构的本质(essence)的一般性的表达式:$\lambda\alpha:*.\alpha\rightarrow\alpha$。这个表达式本身并不是一个类型,而是将类型当作值的一个函数。称之为类型构造子(type constructor)。只有当喂给它类型(比如$\beta$,$\gamma$)时我们可以得到类型:
		
		$(\lambda\alpha:*.\alpha\rightarrow\alpha)\beta\rightarrow_\beta\beta\rightarrow\beta$,
		$(\lambda\alpha:*.\alpha\rightarrow\alpha)\gamma\rightarrow_\beta\gamma\rightarrow\gamma$。
		
		我们由类型$\alpha$抽象出类型$\alpha\rightarrow\alpha$,来获得类型构造子$\lambda\alpha:*.\alpha\rightarrow\alpha$。类似的,还可以构造出更复杂的类型构造子,比如$\lambda\alpha:*.\lambda\beta:*.\alpha\rightarrow\beta$。
		
		而一个显然的问题是:类型构造子的类型是什么?我们可以将$\lambda\alpha:*.\alpha\rightarrow\alpha$看作是一个将类型$\alpha$映射到类型$\alpha\rightarrow\alpha$的函数,因为$\alpha:*$且$\alpha\rightarrow\alpha:*$,我们可以得到:$\lambda\alpha:*.\alpha\rightarrow\alpha:*\rightarrow*$。
		
		因此在*之后,需要一个新的“超级类型(super-type)”,即$*\rightarrow*$。
		
		类似的,我们可以得到:$\lambda\alpha:*.\lambda\beta:*.\alpha\rightarrow\beta:*\rightarrow(*\rightarrow*)$。
		
		需要注意的是,在上一章中,提到了*是所有($\mathbb{T}2$)类型的类型,而$*,*\rightarrow*,...$等类型不属于$\mathbb{T}2$,因此它们的类型也不是*。以及在$\lambda\alpha:*.\alpha\rightarrow\alpha$中,$\alpha\rightarrow\alpha$的类型是*而不是$*\rightarrow*$,是因为$\alpha\rightarrow\alpha$是一个接收类型为$\alpha$的输入值,返回类型为$\alpha$的输出值的函数的类型,而$*\rightarrow*$是一个接收类型,返回类型的函数的类型。
		
		上述扩展称作依赖于类型的类型(types depending on types),扩展后的系统记为$\lambda{\uline{\omega}}$。
			
		所有的超级类型,单独的*以及箭头分割的若干*符号,称为类(kind),所有类的集合$\mathbb{K}$的抽象定义为:
		
		$\mathbb{K}=*|(\mathbb{K}\rightarrow\mathbb{K})$。
		
		而所有类的类型我们使用符号$\square$表示,有且仅有一个的超级超级类型(super-super-type)。也即有$*:\square,*\rightarrow*:\square,...$。
		
		如果$\kappa$是一个类,则对于每个类型是$\kappa$的$M$(也即$M:\kappa$),$M$被称作是一个类型构造子,简称为构造子。而之前的类型,比如$\alpha$或者$\alpha\rightarrow\alpha$也都是构造子,尽管它们什么也没有构造。
		
		我们使用术语(term)真构造子(proper constructor)表示不是类型的构造子(即类型不是*的构造子)。因此构造子的集合被分为了(旧的)类型和真构造子。
		
		最后,使用类别(sort)或符号$s$表示*或$\square$(我认为s其实是代表任意类型的类型或任意构造子类型的类型):
		
		\begin{thm}(构造子,真构造子,类别)
			
			(1) 如果$\kappa:\square$且$M:\kappa$,则$M$是一个构造子,如果$\kappa\not\equiv*$,则$M$是一个真构造子。
			
			(2) 类别($s$)的集合为$\{*,\square\}$。
		\end{thm}
	
		随着$\square$的引入,我们的语法中有四个层级(level):
		
		\begin{thm}(层级,level)
			
			第1层:项(terms);
			
			第2层:构造子(包括类型和真构造子);
			
			第3层:类(kinds);
			
			第4层:超级超级类型$\square$。
		\end{thm}
	
		关于这里的真构造子和类型在同一层,我的理解是,因为真构造子其实就是依赖于类型的类型,正如之前依赖于项的项和项是一个层级的,依赖于类型的类型故也处在类型所在的层级。
	
		对于语句$A:B$,可以得出$B$所处的层级一定比$A$高一级,比如当$A$是一个项时,$B$是一个类型,或者$A$是一个类型时,$B\equiv*$。
		
	\section{$\lambda{\uline{\omega}}$中的类别规则和变量规则,sort-rule and var-rule in $\lambda{\uline{\omega}}$}
		\noindent
		本章中描述的系统叫做$\lambda{\uline{\omega}}$,它是$\lambda{\rightarrow}$的另一个扩展:
		
		\noindent
		- $\lambda{2} = \lambda{\rightarrow}$ 加\ 依赖于类型的项,
		
		\noindent
		- $\lambda{\uline{\omega}} = \lambda{\rightarrow}$ 加\ 依赖于类型的类型。
		
		给出$\lambda{\uline{\omega}}$的具体推导规则。
		
		首先形式化$*$的类型是$\square$,这个规则称为类别规则(sort-rule):
		
		\begin{thm} (类别规则,sort-rule)
			
			(类别,sort) $\emptyset \vdash *:\square$
		\end{thm}
	
		为了确定给定的上下文中所有的声明都是可推导的,在$\lambda{2}$和$\lambda{\uline{\omega}}$中使用变量规则((var)-rule)推导。但是在$\lambda{\uline{\omega}}$中,我们用有一点不同的方式:巧妙地将上下文声明的可推导性与构造合适的上下文相结合。
		
		原因是$\lambda{\uline{\omega}}$中的类型更为复杂,所以必须保证类型的定义是良构的(well-formed)。在$\lambda{\rightarrow}$中,合法类型的集合已经预先给出,所以没有问题,而在$\lambda{2}$中,必须确定一个(合适的)$\lambda{2}$-上下文,这个上下文也提供了在其中使用到的类型需要的条件(requirements),见定义3.4.4(3),$\rho$中出现的所有自由类型变量需要在上下文中声明,此时才可以推定$\rho$的良构与否。因此,与$\lambda{\rightarrow}$不同,$\lambda{2}$中出现在一个推定(judgement)中的类型的合法性不再能通过引用外部的集合来判定,但是应该依赖于包括其上下文的自身推定的一个检查。
		
		在现在的系统中,类型需要的条件更加严格:出现在一个推定中的类型的合法性只取决于(follow)我们是否可以形式化地推导出它。
		
		这个方式是:如果类型$A$已经是合法的,我们只用一个声明$x:A$扩展一个上下文。并且一个语句中的合法类型位于第2层或第3层,也即是一个类型(因为构造子不是类型,不能出现在$:$的右边)或是一个类。可以通过一个规则来表示:
		
		\begin{thm}(变量规则,var-rule)
			
			(变量,var) 如果$x\not\in\Gamma$,则$\dfrac{\Gamma\vdash A:s}{\Gamma,x:A\vdash x:A}$。
		\end{thm}

		需要注意的是,在这条规则中可以看到,$x$的类型不能是$\square$,是因为目前为止,该系统不允许应用时右边是$\square$类型的类(kind),因此也不存在类型是$\square$的绑定变量。

		$s$表示$*$或$\square$,因此$A$是一个类型或一个类(kind),$x$因此也表示一个项变量或是一个类型变量。这个规则允许我们以一个声明$x:A$扩展上下文$\Gamma$,并且在扩展的上下文中推导出与声明一样的语句。

		$x\not\in\Gamma$保证了变量$x$未在$\Gamma$中出现,因此声明在一个上下文中的所有的变量都是不同的,避免了变量名相同(类型不同)时造成的混淆。

		使用(sort)和(var)规则的一个例子,观察它们如何工作:\\

		$\cfrac{\cfrac{(1)\ \emptyset\vdash*:\square}{(2)\ \alpha:*\vdash\alpha:*}\ (var)}{(3)\ \alpha:*,x:\alpha\vdash x:\alpha}\ (var)$\\
		
		第(1)行由(sort)规则推导而出,(1)-(2)以及(2)-(3)都应用了(var)规则。同时可以清楚地发现,新的(var)规则没有$\lambda{\rightarrow}$中的(var)规则通用,因为当前的(var)规则只允许推导出上下文中新添加的最后的声明$x:A$,而$\lambda{\rightarrow}$中,任意出现在$\Gamma$的声明$x:\sigma$都是可推导的。
		
		如在$\lambda\rightarrow$中一样,我们也希望在$\lambda{\uline{\omega}}$中可以推导$\alpha:*,x:\alpha\vdash\alpha:*$,$\alpha:*,\beta:*\vdash\alpha:*$以及$\alpha:*,\beta:*\vdash\beta:*$。但目前尚不可行,因为缺少(var)规则需要的$\bf premiss$,需要注意的是,尽管$\alpha:*,\beta:*\vdash\beta:*$中$\beta:*$是上下文中最后一个声明,但我们不能得到$\bf premiss$:
		
		$\alpha:*\vdash*:\square$。
		
		因为前边所提到的(sort)规则,只给出了空上下文时的规则。
		
		为了解决这个问题,引入了所谓的弱化规则(weakening rule)。
		
	\section{$\lambda{\uline{\omega}}$中的弱化规则(weakening rule)}
		\noindent
		弱化规则允许我们通过添加新的声明来弱化一个推定(judgement)的上下文。
		
		\begin{thm}(弱化规则,weakening rule)
			
			(weak) 如果$x\not\in\Gamma$,则$\dfrac{\Gamma\vdash A:B\ \ \ \Gamma\vdash C:s}{\Gamma,x:C\vdash A:B}$。
		\end{thm}
		
		若上下文$\Gamma$已经可以推导出$A:B$,在$\Gamma$的尾部添加了一个任意的声明(弱化)后仍然可以推导出$A:B$。
		
		需要注意的是第二个$\bf premiss$,添加的声明的类型需要是可推导出的类型或类(kind),与(var)规则类似。
		
		在添加了弱化规则以后,前边所提到的稀疏引理(Thinning Lemma)在$\lambda{\uline{\omega}}$中依然成立,即子上下文可以推定出来的语句在扩展后的上下文中依然可以推定,尽管弱化规则只允许在尾部添加声明。
		
		现在便可以推导$\alpha:*,x:\alpha\vdash\alpha:*$,推导树:\\
		
		$\cfrac{\cfrac{(1)\ \emptyset\vdash*:\square}{(2)\ \alpha:*\vdash\alpha:*}\ (var)\ \ \ \cfrac{(1)\ \emptyset\vdash*:\square}{(2)\ \alpha:*\vdash\alpha:*}\ (var)}{(3)\ \alpha:*,x:\alpha\vdash\alpha:*}\ (weak)$
		
	\section{$\lambda{\uline{\omega}}$中的形成规则(formation rule)}
		\noindent
		在$\lambda{2}$中,有一个叫做(form)的形成规则,用于在一个上下文中类型化语句的构造,这个规则基于$\lambda{2}$-类型的集合$\mathbb{T}2$,正如之前提到的,$\lambda{\uline{\omega}}$中的类型更为复杂,因此,引入了一个“真正”的推导规则,包含$\bf premisses$和$\bf conclusion$,用于类型(以及类(kind))的构造:
		
		\begin{thm}(形成规则,formation rule)\\
			
			(form)\ $\dfrac{\Gamma\vdash A:s\ \ \ \Gamma\vdash B:s}{\Gamma\vdash A\rightarrow B:s}$
		\end{thm}
		
		需要注意的是,规则中出现的三个$s$是相同的,同时表示*或$\square$。
		
		在引入新的(form)规则之前,由(var)规则与(weak)规则只能推导出$*:\square,\alpha:*,x:\alpha$而不能推出箭头类型$*\rightarrow*:\square$以及$\alpha\rightarrow\beta:*$等。引入新的形成规则可以覆盖我们需要的所有类型和类(kind)。\\
		
		例如:$\dfrac{\emptyset\vdash*:\square\ \ \ \emptyset\vdash*:\square}{\emptyset\vdash*\rightarrow*:\square}\ (form)$
		
	\section{$\lambda{\uline{\omega}}$中的应用与抽象规则}
		\noindent
		$\lambda{\uline{\omega}}$中的这两个规则与之前略有不同,首先,用于类型的元变量(meta-variable)的名字不同,因为$\lambda{\uline{\omega}}$中的类型更为通用,其次,需要保证类型是良构的(即可以由上下文推导出):
		
		\begin{thm}(应用规则)\\
			
			(appl) $\dfrac{\Gamma\vdash M:A\rightarrow B\ \ \ \Gamma\vdash N:A}{\Gamma\vdash MN:B}$
		\end{thm}
	
		\begin{thm}(抽象规则)\\
			
			(abst) $\dfrac{\Gamma,x:A\vdash M:B\ \ \ \Gamma\vdash A\rightarrow B:s}{\Gamma\vdash\lambda x:A.M:A\rightarrow B}$
		\end{thm}
	
		需要注意的是,因为$s\in\{*,\square\}$,所以这些规则(包括之前定义的)都同时具有两种作用,比如当$s\equiv*$时,$A\rightarrow B$是一个第二层的类型,如$(\alpha\rightarrow\beta)\rightarrow\gamma$,当$s\equiv\square$时,$A\rightarrow B$就是一个第三层的类型(或者说类,kind),如$(*\rightarrow*)\rightarrow*$。
		
	\section{简化(shortened)推导}
		\noindent
		为了保证推导系统的严谨性以及完整性所定义的这些规则,导致在推导的过程中有许多很无趣以及显而易见的部分,比如在推导$\alpha:*,\beta:*\vdash\alpha\rightarrow\beta:*$时,需要下面这些推定:
	
		$\emptyset\vdash*:\square$ (sort),
		
		$\alpha:*\vdash\alpha:*$ (var),
		
		$\alpha:*\vdash*:\square$ (weak),
		
		$\alpha:*,\beta:*\vdash\alpha:*$ (weak),
		
		$\alpha:*,\beta:*\vdash\beta:*$ (var)。
		
		而上述推定包括$\alpha:*,\beta:*\vdash\alpha\rightarrow\beta:*$都是非常显而易见的。这里不是很有趣的步骤出现在(尤其是)下面三个情况下:
		
		(i) 当使用规则(sort),(var)和(weak)时,
		
		(ii) 当使用规则(form)时,以及
		
		(iii) 当确定(abst)规则的第二个$\bf premiss$的合法性时。
		
		为了将注意力放在真正有趣的步骤上,我们将允许跳过如上的所有推定,或者只确定某个类型的良构与否。因此$\alpha:*,\beta:*\vdash\alpha\rightarrow\beta:*$现在可以直接使用。
		
	\section{变换(conversion)规则}
		\noindent
		变换规则的定义如下:
		
		\begin{thm}(变换规则,conversion rule)\\
			
			(conv) 如果$B=_\beta B'$,则$\dfrac{\Gamma\vdash A:B\ \ \ \Gamma\vdash B':s}{\Gamma\vdash A:B'}$。
		\end{thm}
	
		需要注意的是,因为$B$是作为推定$\Gamma\vdash A:B$中的一个类型,所以$B$已经是良构的。为了保证$B'$也是良构的,添加了第二个$\bf premiss$:$\Gamma\vdash B':s$,保证了$B'$也是良构的类型或类(kind)。
		
		需要注意的是,$\beta$-规约不保证类型匹配,换而言之,$B=_\beta B'$无法保证$B$是良构时$B'$是良构的,比如$\beta\rightarrow\gamma=_\beta(\lambda\alpha:*.\beta\rightarrow\gamma)M$,右边在进行$\beta$-规约时,并不检查$M$的类型,而只是进行符号的替换,当$M$的类型不是*时,右边即不是良构的。
		
		在拥有了变换规则之后,我们可以进行如下推导,其中令$\Gamma\equiv\beta:*,x:(\lambda\alpha:*.\alpha\rightarrow\alpha)\beta$:
		
		$\cfrac{\Gamma\vdash x:(\lambda\alpha:*.\alpha\rightarrow\alpha)\beta\ \ \ \Gamma\vdash\beta\rightarrow\beta:*}{\Gamma\vdash x:\beta\rightarrow\beta}$ (conv)
		
		需要注意的是,推定中对象(subject),即$A:B$中的$A$进行规约后的类型不变,且仍可被推导出,而这一定理可以由之前的规则推出,不再赘述。
		
		至此,所有的$\lambda{\uline{\omega}}$-规则如下:
		
		\fbox{\shortstack[l]{
			\\(sort) $\emptyset \vdash *:\square$\\
			(var) 如果$x\not\in\Gamma$,则$\cfrac{\Gamma\vdash A:s}{\Gamma,x:A\vdash x:A}$\\
			(weak) 如果$x\not\in\Gamma$,则$\cfrac{\Gamma\vdash A:B\ \ \ \Gamma\vdash C:s}{\Gamma,x:C\vdash A:B}$\\
			(form)\ $\cfrac{\Gamma\vdash A:s\ \ \ \Gamma\vdash B:s}{\Gamma\vdash A\rightarrow B:s}$\\
			(appl) $\cfrac{\Gamma\vdash M:A\rightarrow B\ \ \ \Gamma\vdash N:A}{\Gamma\vdash MN:B}$\\
			(abst) $\cfrac{\Gamma,x:A\vdash M:B\ \ \ \Gamma\vdash A\rightarrow B:s}{\Gamma\vdash\lambda x:A.M:A\rightarrow B}$\\
			(conv) 如果$B=_\beta B'$,则$\cfrac{\Gamma\vdash A:B\ \ \ \Gamma\vdash B':s}{\Gamma\vdash A:B'}$
		}}
	
	\section{$\lambda{\uline{\omega}}$的性质}
		\noindent
		$\lambda{\uline{\omega}}$满足之前几章所提到的大部分性质,但是类型唯一性引理需要进行调整:
		
		\begin{lemma}(类型唯一性引理)
			
			如果$\Gamma\vdash A:B_1$且$\Gamma\vdash A:B_2$,则$B_1=_\beta B_2$。
		\end{lemma}
\end{document}
