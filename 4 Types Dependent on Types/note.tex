\documentclass[UTF8]{article}
\usepackage{ctex}
\usepackage{amssymb}
\usepackage{amsmath}
\usepackage{graphicx}
\newtheorem{thm}{定义}[section]
\newtheorem{notation}[thm]{记号}
\newtheorem{lemma}[thm]{引理}

\title{依赖于类型的类型\\[2ex]\begin{large}读书笔记\end{large}}
\author{许博}
\date{}

\begin{document}
\maketitle
	\section{类型构造子}
		上一章中引入了一般性的(generalised)项,通过从类型变量抽象项。比如作用于确定类型$\sigma$的恒等函数$\lambda x:\sigma.x$可以被泛化成项$\lambda\alpha:*.\lambda x:\alpha.x$,多态的(polymorphic)恒等函数,抽象于$\alpha$。
		
		自然而然的,我们也希望可以通过类似的方式构造一般性的类型。比如形如$\beta\rightarrow\beta,\gamma\rightarrow\gamma,...$,等具有结构$\lozenge\rightarrow\lozenge$的类型,其中箭头的左边和右边是一样的类型。(我认为这里如果延续之前的符号使用方式,我认为$\beta$可能使用$\sigma$等替换更为合适,$\beta\rightarrow\beta$更像是抽象于$\beta$的类型,而非一个具体的类型,此时$\beta$甚至与$\lozenge$是没有区别的)。
		
		为了处理这种情况,我们引入一个包含了这种结构的本质(essence)的一般性的表达式:$\lambda\alpha:*.\alpha\rightarrow\alpha$。这个表达式本身并不是一个类型,而是将类型当作值的一个函数。称之为类型构造子(type constructor)。只有当喂给它类型(比如$\beta$,$\gamma$)时我们可以得到类型:
		
		$(\lambda\alpha:*.\alpha\rightarrow\alpha)\beta\rightarrow_\beta\beta\rightarrow\beta$,
		$(\lambda\alpha:*.\alpha\rightarrow\alpha)\gamma\rightarrow_\beta\gamma\rightarrow\gamma$。
		
		我们由类型$\alpha$抽象出类型$\alpha\rightarrow\alpha$,来获得类型构造子$\lambda\alpha:*.\alpha\rightarrow\alpha$。类似的,还可以构造出更复杂的类型构造子,比如$\lambda\alpha:*.\lambda\beta:*.\alpha\rightarrow\beta$。
		
		而一个显然的问题是:类型构造子的类型是什么?
		
\end{document}
