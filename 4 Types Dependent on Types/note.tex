\documentclass[UTF8]{article}
\usepackage{ctex}
\usepackage{amssymb}
\usepackage{amsmath}
\usepackage{graphicx}
\newtheorem{thm}{定义}[section]
\newtheorem{notation}[thm]{记号}
\newtheorem{lemma}[thm]{引理}

\title{依赖于类型的类型\\[2ex]\begin{large}读书笔记\end{large}}
\author{许博}
\date{}

\begin{document}
\maketitle
		本章将介绍另一个扩充$\lambda{\rightarrow}$的系统$\lambda{\underline{\omega}}$,而非在$\lambda{2}$上继续扩充的系统。

	\section{类型构造子}
		上一章中引入了一般性的(generalised)项,通过从类型变量抽象项。比如作用于确定类型$\sigma$的恒等函数$\lambda x:\sigma.x$可以被泛化成项$\lambda\alpha:*.\lambda x:\alpha.x$,多态的(polymorphic)恒等函数,抽象于$\alpha$。
		
		通过类似的方式也可以构造一般性的类型。比如形如$\beta\rightarrow\beta,\gamma\rightarrow\gamma,...$,等具有结构$\lozenge\rightarrow\lozenge$的类型,其中箭头的左边和右边是一样的类型。(我认为这里如果延续之前的符号使用方式,我认为$\beta$可能使用$\sigma$等替换更为合适,$\beta\rightarrow\beta$更像是抽象于$\beta$的类型,而非一个具体的类型,此时$\beta$甚至与$\lozenge$是没有区别的)。
		
		为了处理这种情况,引入一个包含了这种结构的本质(essence)的一般性的表达式:$\lambda\alpha:*.\alpha\rightarrow\alpha$。这个表达式本身并不是一个类型,而是将类型当作值的一个函数。称之为类型构造子(type constructor)。只有当喂给它类型(比如$\beta$,$\gamma$)时我们可以得到类型:
		
		$(\lambda\alpha:*.\alpha\rightarrow\alpha)\beta\rightarrow_\beta\beta\rightarrow\beta$,
		$(\lambda\alpha:*.\alpha\rightarrow\alpha)\gamma\rightarrow_\beta\gamma\rightarrow\gamma$。
		
		我们由类型$\alpha$抽象出类型$\alpha\rightarrow\alpha$,来获得类型构造子$\lambda\alpha:*.\alpha\rightarrow\alpha$。类似的,还可以构造出更复杂的类型构造子,比如$\lambda\alpha:*.\lambda\beta:*.\alpha\rightarrow\beta$。
		
		而一个显然的问题是:类型构造子的类型是什么?我们可以将$\lambda\alpha:*.\alpha\rightarrow\alpha$看作是一个将类型$\alpha$映射到类型$\alpha\rightarrow\alpha$的函数,因为$\alpha:*$且$\alpha\rightarrow\alpha:*$,我们可以得到:$\lambda\alpha:*.\alpha\rightarrow\alpha:*\rightarrow*$。
		
		因此在*之后,需要一个新的“超级类型(super-type)”,即$*\rightarrow*$。
		
		类似的,我们可以得到:$\lambda\alpha:*.\lambda\beta:*.\alpha\rightarrow\beta:*\rightarrow(*\rightarrow*)$。
		
		需要注意的是,在上一章中,提到了*是所有($\mathbb{T}2$)类型的类型,而$*,*\rightarrow*,...$等类型不属于$\mathbb{T}2$,因此它们的类型也不是*。以及在$\lambda\alpha:*.\alpha\rightarrow\alpha$中,$\alpha\rightarrow\alpha$的类型是*而不是$*\rightarrow*$,是因为$\alpha\rightarrow\alpha$是一个接收类型为$\alpha$的输入值,返回类型为$\alpha$的输出值的函数的类型,而$*\rightarrow*$是一个接收类型,返回类型的函数的类型。
		
		上述扩展称作依赖于类型的类型(types depending on types),扩展后的系统记为$\lambda{\underline{\omega}}$。
			
		所有的超级类型,单独的*以及箭头分割的若干*符号,称为类(kind),所有类的集合$\mathbb{K}$的抽象定义为:
		
		$\mathbb{K}=*|(\mathbb{K}\rightarrow\mathbb{K})$。
		
		而所有类的类型我们使用符号$\square$表示,有且仅有一个的超级超级类型(super-super-type)。也即有$*:\square,*\rightarrow:\square,...$。
		
		如果$\kappa$是一个类,则对于每个类型是$\kappa$的$M$(也即$M:\kappa$),$M$被称作是一个类型构造子,简称为构造子。而之前的类型,比如$\alpha$或者$\alpha\rightarrow\alpha$也都是构造子,尽管它们什么也没有构造。
		
		我们使用术语(term)真构造子(proper constructor)表示不是类型的构造子(即类型不是*的构造子)。因此构造子的集合被分为了(旧的)类型和真构造子。
		
		最后,使用类别(sort)或符号$s$表示*或$\square$:
		
		\begin{thm}(构造子,真构造子,类别)
			
			(1) 如果$\kappa:\square$且$M:\kappa$,则$M$是一个构造子,如果$\kappa\not\equiv*$,则$M$是一个真构造子。
			
			(2) 类别($s$)的集合为$\{*,\square\}$。
		\end{thm}
	
		随着$\square$的引入,我们的语法中有四个层级(level):
		
		\begin{thm}(层级,level)
			
			第1层:项(terms);
			
			第2层:构造子(包括类型和真构造子);
			
			第3层:类(kinds);
			
			第4层:超级超级类型$\square$。
		\end{thm}
			
\end{document}
