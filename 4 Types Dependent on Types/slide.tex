\documentclass[UTF8,aspectratio=169,mathserif]{beamer}
\usepackage{ctex}
\usepackage{ulem}
\usepackage{color}
\usepackage{amssymb}
\usepackage{amsmath}
\usetheme{Berlin}
\setbeamertemplate{navigation symbols}{}

\title{依赖于类型的类型}
\subtitle{Types Depending on Types}
\author{报告人:许博}
\date{}

\begin{document}
	
	\begin{frame}
		\titlepage
	\end{frame}

	\begin{frame}{目录}
		\tableofcontents
	\end{frame}

	\section{类型构造子}
	\begin{frame}{类型构造子}
		\begin{block}{$\lambda{\rightarrow}$}
			$\lambda x:\alpha.x$
		\end{block}
		
		\begin{block}{$\lambda{2},\ \lambda{\rightarrow}\ + $ terms depending on types}
			$\lambda\alpha:*.\lambda x:\alpha.x$
			
			$(\lambda\alpha:*.\lambda x:\alpha.x)\beta\rightarrow_\beta\lambda x:\beta.x$
		\end{block}
	
		\begin{block}{$\lambda{\underline{\omega}},\ \lambda{\rightarrow}\ + $ types depending on types}
			$\beta\rightarrow\beta,\gamma\rightarrow\gamma,...$ 等具有结构$\lozenge\rightarrow\lozenge$的类型,其中箭头的左右是一样的类型
			
			\textcolor{red}{$\lambda\alpha:*.\alpha\rightarrow\alpha$,类型构造子}
			
			$(\lambda\alpha:*.\alpha\rightarrow\alpha)\beta\rightarrow_\beta\beta\rightarrow\beta$
		\end{block}
	\end{frame}

	\begin{frame}{类型构造子的类型,超级类型}
		\begin{block}{$\lambda\alpha:*.\alpha\rightarrow\alpha$的类型}
			$\alpha:*$
			
			$\alpha\rightarrow\alpha:*$
			
			$\lambda\alpha:*.\alpha\rightarrow\alpha:*\rightarrow*$
			
			$\lambda\alpha:*.\lambda\beta:*.\alpha\rightarrow\beta:*\rightarrow(*\rightarrow*)$
		\end{block}
		\begin{block}{超级类型,记为类(kind),$\mathbb{K}$}
			$*, *\rightarrow*, *\rightarrow(*\rightarrow*), ...$
			
			$\mathbb{K}=*|(\mathbb{K}\rightarrow\mathbb{K})$
		\end{block}
		\begin{block}{超级超级类型,$\square$}
			$*:\square, *\rightarrow*:\square, *\rightarrow(*\rightarrow*):\square, ...$
		\end{block}
	\end{frame}

	\section{类别和变量规则}
	\begin{frame}
		$\lambda{\underline{\omega}}$
	\end{frame}

	\section{弱化规则(weakening rule)}
	\begin{frame}
		
	\end{frame}

	\section{形成规则(formation rule)}
	\begin{frame}
		
	\end{frame}

	\section{应用与抽象规则}
	\begin{frame}
		
	\end{frame}

	\section{简化(shortened)推导}
	\begin{frame}
		
	\end{frame}

	\section{变换(conversion)规则}
	\begin{frame}
		
	\end{frame}

	\section{性质}
	\begin{frame}
		
	\end{frame}
	
\end{document}