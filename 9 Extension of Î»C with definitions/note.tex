\documentclass[UTF8]{article}
\usepackage{ctex}
\usepackage{ulem}
\usepackage{amssymb}
\usepackage{amsmath}
\usepackage{graphicx}
\newtheorem{thm}{定义}[section]
\newtheorem{notation}[thm]{记号}
\newtheorem{lemma}[thm]{引理}

\makeatletter
\newcommand{\rmnum}[1]{\romannumeral #1}
\newcommand{\Rmnum}[1]{\expandafter\@slowromancap\romannumeral #1@}
\makeatother

\title{9 以定义扩展$\lambda{\rm C}$\\Extension of $\lambda{\rm C}$ with definitions\\[2ex]\begin{large}读书笔记\end{large}}
\author{许博}
\date{}

\begin{document}
\maketitle
	\section{$\lambda{\rm C}$扩展到系统$\lambda{\rm D_0}$}
	\noindent
	本章在$\lambda{\rm C}$的基础上扩展通常意义上定义的形式化版本,也即所谓的描述性定义(descriptive definitions)。扩展后的系统$\lambda{\rm D_0}$尚不能完全支持公理以及公理概念的表示,相应的扩展会在下一章引入$\lambda{\rm D}$时说明。
	
		为给出$\lambda{\rm D_0}$的合适的描述,首先扩展表达式的集合。$\lambda{\rm D_0}$中的表达式与$\lambda{\rm D}$中相同,因此记集合为$\mathcal{E}_{\lambda{\rm D}}$。
		
		假设除了之前定义的变量集合$V$以外,还有常量的集合$C$。使用符号$a,a_1,a_i,a',b,...$作为常量的名字,正如我们使用$x,x_1,x_i,x',y,...$作为变量的名字一样。另外,还假设变量和常量来自不相交的集合,而$*$和$\square$是特殊符号,不属于$V$和$C$:
		
		$V\cap C=\emptyset,\ *\not=\square,\ *,\square\notin V\cup C$
		
		\begin{thm}($\mathcal{E}_{\lambda{\rm D}}$)
			
			$\mathcal{E}_{\lambda{\rm D}}=V|\square|*|(\mathcal{E}_{\lambda{\rm D}}\mathcal{E}_{\lambda{\rm D}})|(\lambda V:\mathcal{E}_{\lambda{\rm D}}.\mathcal{E}_{\lambda{\rm D}})|(\Pi V:\mathcal{E}_{\lambda{\rm D}}.\mathcal{E}_{\lambda{\rm D}})|C(\overline{\mathcal{E}_{\lambda{\rm D}}})$
		\end{thm}
	
		其中$\overline{\mathcal{E}_{\lambda{\rm D}}}$中的上划线表示这是一个$\mathcal{E}_{\lambda{\rm D}}$-表达式的列表。
		
		引入“环境,environment”表示一个定义的列表。
		
		\begin{thm}($\lambda{\rm D_0}$中的描述性定义;环境)
			
			\noindent
			(1) 在$\mathcal{E}_{\lambda{\rm D}}$中,一个(描述性)定义具有形式
			
			$\overline{x}:\overline{A}\triangleright a(\overline{x}):=M:N$
			
			\noindent
			其中所有的$x_i\in V,a\in C$,并且所有的$A_i,M,N\in\mathcal{E}_{\lambda{\rm D}}$
			
			\noindent
			(2) 一个环境$\Delta$是一个有限(空或非空)的定义列表。
		\end{thm}
	
		使用诸如$\mathcal{D},\mathcal{D}_i,...$等符号作为元名称表示定义。一个长度为$k$的环境可以被表示为如$\Delta\equiv\mathcal{D}_1,...,\mathcal{D}_k$。
		
		关于定义,区分以下元素:
		
		\begin{thm}(定义中的元素)
			
			\noindent
			令$\mathcal{D}\equiv\overline{x}:\overline{A}\triangleright a(\overline{x}):=M:N$是一个定义。则:\\
			- $\overline{x}:\overline{A}$是$\mathcal{D}$中的上下文\\
			- $a$是$\mathcal{D}$中被定义的常量,$\overline{x}$是参数列表\\
			- $a(\overline{x})$是$\mathcal{D}$中的$definiendum$\\
			- $M:N$是$\mathcal{D}$中的语句,$M$是$definiens$或$\mathcal{D}$的主体,$N$是$\mathcal{D}$的类型。
		\end{thm}
\end{document}
