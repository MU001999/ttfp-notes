\documentclass[UTF8]{article}
\usepackage{ctex}
\usepackage{ulem}
\usepackage{dsfont}
\usepackage{amssymb}
\usepackage{amsmath}
\usepackage{graphicx}
\newtheorem{thm}{定义}[section]
\newtheorem{notation}[thm]{记号}
\newtheorem{lemma}[thm]{引理}

\makeatletter
\newcommand{\rmnum}[1]{\romannumeral #1}
\newcommand{\Rmnum}[1]{\expandafter\@slowromancap\romannumeral #1@}
\makeatother
\newcommand{\dperp}{\perp\!\!\!\perp}

\title{15/16章读书笔记}
\author{许博}
\date{}

\begin{document}
\maketitle

\newpage
\section{15 An Elaborated Example}
	证明略
	
\newpage
\section{16 Further Perspectives}
	\subsection{$\lambda{\rm D}$的应用}
	\noindent
	总结类型理论(尤其是$\lambda{\rm D}$)作为一个用于形式化数学的系统时的主要特性:
	
		\textit{通过类型理论的数学形式化,formalisation of mathematics via type theory}
		
		\textit{数学的检查,checking of mathematics} 可以检查不完备的证明或者使用了不合法的逻辑步骤的证明等。
		
		\textit{证明发展,proof development} 可以构建推理的步骤,也即证明的逐步发展,对于开始学习逻辑和数学的学生尤其有帮助。
		
		\textit{库,libraries} 通过命名定义与证明,可以得到一个巨大的环境,也即包含了数学概念和定理的定义的形式化的数学的库。
\end{document}
