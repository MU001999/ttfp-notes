\documentclass[UTF8]{article}
\usepackage{ctex}
\usepackage{ulem}
\usepackage{amssymb}
\usepackage{amsmath}
\usepackage{graphicx}
\newtheorem{thm}{定义}[section]
\newtheorem{notation}[thm]{记号}
\newtheorem{lemma}[thm]{引理}

\makeatletter
\newcommand{\rmnum}[1]{\romannumeral #1}
\newcommand{\Rmnum}[1]{\expandafter\@slowromancap\romannumeral #1@}
\makeatother

\title{$\lambda{\rm C}$中逻辑概念的编码\\The encoding of logical notions in $\lambda{\rm C}$\\[2ex]\begin{large}读书笔记\end{large}}
\author{许博}
\date{}

\begin{document}
\maketitle
	\section{类型理论中的谬论(absurdity)与否定(negation)}
	\noindent
	在章节 5.4 中,通过编码蕴含式$A\Rightarrow B$为函数类型$A\rightarrow B$,模拟蕴含式的行为,包括它的导入和消解规则。因为$\lambda{\rm P}$是$\lambda{\rm C}$的一部分,所以$\lambda{\rm C}$中同样拥有最小谓词逻辑。
	
		本章将处理更多的接词(connective),比如否定($\neg$),合取($\land$)和析取($\lor$)。这些在$\lambda{\rm P}$中不能表示,但在$\lambda{\rm C}$中存在非常优雅的方式去编码这些概念。
		
		将否定$\neg A$看作蕴含式$A\Rightarrow \bot$,其中$\bot$是“谬论(absurdity)”,也可以称为“矛盾(contradiction)”。因此$\neg A$被解释为“$A$蕴含了谬论”。为了这个目标,我们需要谬论的编码:
		
	\noindent
	\textit{\Rmnum{1}. 谬论,Absurdity}\\
	命题“谬论”或$\bot$的一个独特的性质是:如果$\bot$为真,则每一个命题都为真。
\end{document}
