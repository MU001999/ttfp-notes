\documentclass[UTF8]{article}
\usepackage{ctex}
\usepackage{ulem}
\usepackage{amssymb}
\usepackage{amsmath}
\usepackage{graphicx}
\newtheorem{thm}{定义}[section]
\newtheorem{notation}[thm]{记号}
\newtheorem{lemma}[thm]{引理}

\makeatletter
\newcommand{\rmnum}[1]{\romannumeral #1}
\newcommand{\Rmnum}[1]{\expandafter\@slowromancap\romannumeral #1@}
\makeatother

\title{前六章的总结}
\author{许博}
\date{}

\begin{document}
\maketitle
	第一章从无类型的$\lambda$演算的说明开始,$\lambda$演算是以函数的行为作为基础的演算,包括变量绑定和替换。并且形式化定义了无类型$\lambda$演算,包括它的构造规则以及求值规则等,诸如$\alpha$变换以及$\beta$规约等也都进行了定义。除此了无类型$\lambda$演算的形式化规则以外,第一章还讨论了这个系统所满足的一些定理和引理,比如不动点定理等等。最后列出了无类型的$\lambda$演算的积极和消极的方面,积极的方面比如无类型的$\lambda$演算是图灵完备的,消极的方面比如允许一些没有意义的项的出现等等。
	
	而在第一章引出了无类型的$\lambda$演算之后,无类型的$\lambda$演算具有一些缺陷,而这些缺陷基本上都是由于其过于自由导致的,因此引出了类型,为这个系统添加一些限制,类型也在本书的剩余部分起到重要的作用(毕竟书名就是类型理论)。第二章到第六章的部分,每一章都会在之前的系统上进行扩展,以及添加相对应的规则,然后引出一个新的推导系统,这些系统适用的地方不同,所具有的特点也有不同,其中第六章并未引入新的概念,而是将之前引出的系统组合成一个系统。
	
	第二章引出了简单类型的$\lambda$演算$\lambda{\rightarrow}$,它是在无类型的$\lambda$演算的基础上扩展了简单类型,约束了函数的绑定变量的类型,使得在应用时的应用的左右类型相互匹配。本章对只对项的类型做了规定,但在此基础上,引出了类型理论中需要解决的三种问题:Well-typedness, Type Checking 和 Term Finding。除此之外还引出了PAT-解释,将$\lambda$演算与逻辑证明相联系。
	
	第三章在$\lambda{\rightarrow}$的基础上扩展了依赖于类型的项,引出了系统$\lambda{2}$,依赖于类型的项可以用来构造不指定具体类型的项。由于类型和项并不处于同一阶,因此$\lambda{2}$中存在二阶抽象和应用,并且增加了$\Pi$类型以表示在类型中可能存在的自由类型变量。
	
	第四章在$\lambda{\rightarrow}$(而非$\lambda{2}$)的基础上扩展了依赖于类型的类型,引出了系统$\lambda{\uline{\omega}}$,在本章中引入了类型构造子以及种类(kind)的概念,对类型的结构进行了抽象。$\lambda{\uline{\omega}}$中依赖于类型的类型和类型处于同一阶,因此不存在$\Pi$类型。
	
	第五章在$\lambda{\rightarrow}$的基础上扩展了依赖于项的类型,引出了系统$\lambda{\rm P}$。$\lambda{\rm P}$允许我们形式化谓词,并且根据PAT-解释,可以通过$\lambda{\rm P}$构建出最小谓词逻辑。
	
	第六章将第二章到第五章引出的系统综合,引入了$\lambda$立方体的概念,得到一个具有以上扩展的系统$\lambda{\rm C}$,综合了上述系统的特点。而$\lambda{\rm C}$作为一个大而全(相对于之前的系统)的推导系统,也是本书剩余部分的基础。
\end{document}
