\documentclass[UTF8]{article}
\usepackage{ctex}
\usepackage{amssymb}
\usepackage{amsmath}
\usepackage{graphicx}
\newtheorem{thm}{定义}[section]
\newtheorem{notation}[thm]{记号}
\newtheorem{lemma}[thm]{引理}

\title{简单类型化的$\lambda$-演算\\[2ex]\begin{large}读书笔记\end{large}}
\author{许博}
\date{}

\begin{document}
\maketitle
	\section{引入类型}
		第一章中介绍的无类型$\lambda$-演算的定义十分简洁而且优雅,但是有些时候过于自由,比如它允许$(xx)$这样没有意义的$\lambda$-项。为了更好地掌握函数的期望行为,本章将引入类型。
		
		函数通常被认为作用于某一集合的对象上,比如自然数的集合或者一条线上点的集合。
		
		类型的添加提供了在允许的输入值上的一些限制,比如定义在自然数域上的函数只能接受自然数作为输入值,即使对于非法的输入值的可能的输出值是清楚的。比如定义在自然数域上的立方函数,不能计算 3.5 的立方值,只能定义另一个作用于更大的域,比如实数域的立方函数。
		
		本章中引入的简单类型形式化了第一个重要的步骤,尽管某些时候限制性过强:我们不能通过简单类型来表示足够多的函数。后续章节将会增加更多的类型来加强系统的表达能力。
		
	\section{简单类型}
		添加(抽象)类型的一个直接的方式是从一个类型变量的无限集合出发,然后添加乘法规则来构建更复杂的类型,即函数类型。
		
		记类型变量的无限集合:$\mathbb{V} = \{\alpha, \beta, \gamma, ...\}$。
		
		\begin{thm} 所有简单类型的集合$\mathbb{T}$
			
			(1)(类型变量)如果$\alpha \in \mathbb{V}$,则 $\alpha \in \mathbb{V}$。
			
			(2)(箭头类型)如果$\sigma,\tau\in\mathbb{T}$,则$(\sigma\rightarrow\tau)\in\mathbb{T}$。
			
		\end{thm}
	
		抽象语法描述为:$\mathbb{T}=\mathbb{V}|\mathbb{T}\rightarrow\mathbb{T}$。
		
		\begin{notation} 
			(1) 希腊字母$\alpha,\beta,...$以及它们的变体用于表示$\mathbb{V}$中的类型变量。
			
			(2) 使用 $\sigma,\tau,...$(偶尔使用$A,B,...$)来表示任意的简单类型。
			
			(3) 最外层的括号会被省略。
			
			(4) 箭头类型中的括号是右结合的。
		\end{notation}
	
		$\alpha_1 \rightarrow \alpha_2 \rightarrow \alpha_3 \rightarrow \alpha_4$ 是 $(\alpha_1 \rightarrow (\alpha_2 \rightarrow (\alpha_3 \rightarrow \alpha_4)))$ 的简写。
		
		使用形如 $M:\sigma$ 的格式表示项 M 具有类型 $\sigma$。
		
		假设对于每个类型$\sigma$,都有无限的变量可以获得,而每一个变量都只有一个唯一的类型:如果$x:\sigma \land x:\tau$,则$\sigma \equiv \tau$。
		
		类型化应用和抽象所需要的必要条件如下:
		
		(1)(应用):如果 $M:\sigma\rightarrow\tau\land N:\sigma$,则$MN:\tau$。
		
		(2)(抽象):如果$x:\sigma\land M:\tau$,则$\lambda x.M:\sigma\rightarrow\tau$。
		
		\begin{thm}
			(可类型化的项,typable term)如果存在类型$\sigma$使得$M:\sigma$,则项 M 称为可类型化的(typable)。
		\end{thm}

	\section{邱奇-类型化(Church-typing)和柯里-类型化(Curry-typing)}
		类型化一个$\lambda$-项从类型化它的变量开始,有两种方式给出变量的类型:
		
		(1) 在每一个变量的声明中,显式的指定类型。这种方式叫做显式类型化。如果考虑到在确定应用的类型时的限制,更复杂的项的类型也可以直接确定(我猜这里的意思是假设所有的应用中左右项的类型都是满足限制的)。
		
		(2) 另一种方式不给出变量的类型,在一定范围内隐式的确定变量的类型。这种方式叫做隐式类型化,通过一个查找过程来推断类型,其中可能包含了猜测。
		
		为了清楚的表示,约束变量的类型会在$\lambda$后引入它们(对应的绑定变量)时直接标记,自由变量的类型则会在所谓的上下文(context)中给出,顺序随意。
		
		如 $x:\alpha,y:(\alpha\rightarrow\alpha)\rightarrow\beta\vdash(\lambda z:\beta.\lambda u:\gamma.z)(yx):\gamma\rightarrow\beta$,读作在上下文 $x:\alpha,y:(\alpha\rightarrow\alpha)\rightarrow\beta$ 中,项 $(\lambda z:\beta.\lambda u:\gamma.z)(yx):\gamma\rightarrow\beta$ 具有类型 $\gamma\rightarrow\beta$。分隔符 $\vdash$ 分割上下文和可类型化的项。
		
	\section{$\lambda{\rightarrow}$的推导规则}
		现在$\lambda$-项具有类型信息,给出预先类型化的$\lambda$-项的新的定义,新的集合也被记为$\Lambda_\mathbb{T}$:
		
		\begin{thm}(预先类型化的$\lambda$-项,$\Lambda_\mathbb{T}$)
			
			$\Lambda_\mathbb{T} = V|(\Lambda_\mathbb{T}\Lambda_\mathbb{T})|(\lambda V:\mathbb{T}.\Lambda_\mathbb{T})$
		\end{thm}
		
		为了表示形如$\lambda$-项 M 在上下文$\Gamma$中具有类型$\sigma'$,进行定义:
		
		\begin{thm}(语句,声明,上下文,判决)
			
			(1) 一个语句形如 $M:\sigma$,其中$M\in\Lambda_\mathbb{T} \land\sigma\in\mathbb{T}$。
			
			(2) 一个声明是一个变量作为主体的语句。
			
			(3) 一个上下文是具有不同主体的声明的列表。
			
			(4) 一个判决形如 $\Gamma\vdash M:\sigma$,其中$\Gamma$是一个上下文,$M:\sigma$是一个语句。
		\end{thm}
	
		给出能够判断一个判决$\Gamma\vdash M$是否是可推导的,也即 M 在上下文$\Gamma$中具有类型$\sigma$的形式化规则。
		
		给出的规则形式是一个所谓的推导系统(derivation system):每一个规则解释某种判决如何能够被形式化确定。三个推导规则中的每一个都是所谓的前提-结论格式,一些前提出现在水平线之上,结论在水平线下边:\\
		
			$\bf\dfrac{premiss\ 1\ \ premiss\ 2\ \ ...\ \ premiss\ n}{conclusion}$\\
		
		这个推导结构的含义是:在已知$\bf premiss\ 1,\ premiss\ 2,\ ...\ ,premiss\ n$的情况下。可以推出$\bf conclusion$。当前提的个数为 0 时,则只需要写结论即可,同时无需水平线。
		
		给出$\lambda{\rightarrow}$的三条推导规则,这三条规则为$\lambda{\rightarrow}$形式化了一个推导系统:
		
		\begin{thm} $\lambda{\rightarrow}$的推导规则
			
			(变量)如果$x:\sigma\in\Gamma$,则$\Gamma\vdash x:\sigma$\\
			
			(应用)$\dfrac{\Gamma\vdash M:\sigma\rightarrow\tau\ \ \Gamma\vdash N:\sigma}{\Gamma\vdash MN:\tau}$\\
			
			(抽象)$\dfrac{\Gamma,x:\sigma\vdash M:\tau}{\Gamma\vdash\lambda x:\sigma.M:\sigma\rightarrow\tau}$\\
			
		\end{thm}
	
		\begin{thm} 合法的$\lambda{\rightarrow}$-项
			
			如果存在上下文$\Gamma$以及类型$\rho$使得$\Gamma\vdash M:\rho$,则在$\lambda{\rightarrow}$中的预先类型化的项 M 称为合法的。
		\end{thm}
	
	\section{$\lambda{\rightarrow}$中推导的不同格式}
		树型,线型以及旗型(一种特殊的线型,更加精简)。
		
	\section{类型理论中需要解决的问题的种类}
		三种:
		
		(1) Well-typedness,也叫 Typability。我的理解是良好的类型定义,类型定义的信息是足够的:
		
		$?\vdash term : ?$.
		
		如果一个项是合法的,能够找到一个合适的上下文和项的类型,如果不是合法的,也能展示出为什么是错误的。
		
		(1a) (1) 的一个变体也即 $Type Assignment$,类型赋值,在给出上下文的时候,找到项的类型:
		
		$context \vdash term: ?$.
		
		(2) 类型检查:
		
		$context\vdash^? term:type$,在给定上下文,项以及它的类型时,检查在给定的上下文时,给定的项是否具有给定的类型。
		
		(3) 项的查找:
		
		$context\vdash ? : type$.
		
		给定上下文和项的类型,找到是否存在一个项在给定的上下文中具有给定的类型。

	\section{项的查找}
		PAT-解释,PAT 意为 propositions-as-types 以及 proofs-as-terms,查找在上下文中符合给定类型的$\lambda{\rightarrow}$-项,整个过程可以解释为等价的逻辑证明。
		
	\section{$\lambda{\rightarrow}$的一些性质}
		首先是一些定义:
		
		\begin{thm} (域,dom,子上下文,$\subseteq$,排列,投影,$\upharpoonright$)
			
			(1) 如果 $\Gamma\equiv x_1:\sigma_1,...,x_n:\sigma_n$,则$\Gamma$的域或$dom(\Gamma)$是列表$(x_1,...,x_n)$。
			
			(2) 如果所有出现在$\Gamma'$的声明以同样的顺序出现在$\Gamma$中,则上下文$\Gamma'$是$\Gamma$的一个子上下文,或$\Gamma'\subseteq\Gamma$。
			
			(3) 如果所有出现在$\Gamma'$的声明出现在$\Gamma$中,如果所有出现在$\Gamma$的声明也出现在$\Gamma'$中,则$\Gamma'$是$\Gamma$的一个排列(permutation),反之亦然。
			
			(4) 如果 $\Gamma$ 是一个上下文,且 $\Phi$ 是变量的一个集合,则$\Gamma$在$\Phi$上的投影,或记为$\Gamma\upharpoonright\Phi$,是$\Gamma$的子上下文,记为$\Gamma'$,满足条件$dom(\Gamma') = dom(\Gamma)\cap\Phi$。
		\end{thm}
	
		性质:
		
		\begin{lemma} (自由变量引理)
			
			如果 $\Gamma\vdash L:\sigma$,则 $FV(L)\subseteq dom(\Gamma)$。
		\end{lemma}
	
		\begin{lemma} (Thinning,Condensing,Permutation)
			
			(1) (稀疏引理,Thinning)令上下文$\Gamma'$和$\Gamma''$满足条件$\Gamma'\subseteq\Gamma''$,如果$\Gamma'\vdash M:\sigma$,也有$\Gamma''\vdash M:\sigma$。
			
			(2) (压缩引理,Condensing)如果$\Gamma\vdash M$,也有$\Gamma\upharpoonright FV(M) \vdash M:\sigma$。
			
			(3) (排列引理,Permutation)如果$\Gamma\vdash M:\sigma$且$\Gamma'$是$\Gamma$的一个排列,则$\Gamma'$也是一个上下文且$\Gamma'\vdash M:\sigma$。
		\end{lemma}
	
		\begin{lemma} (生成引理,Generation Lemma)
			
			(1) 如果 $\Gamma\vdash x:\sigma$,则 $x:\sigma \in\Gamma$。
			
			(2) 如果 $\Gamma\vdash M N:\tau$,则存在类型$\sigma$使得$\Gamma\vdash M:\sigma \rightarrow \tau$且$\Gamma\vdash N:\sigma$。
			
			(3) 如果 $\Gamma\vdash\lambda x:\sigma.M:\rho$,则存在$\tau$使得$\Gamma,x:\sigma\vdash M:\tau$且$\rho\equiv\sigma\rightarrow\tau$。
		\end{lemma}
	
		生成引理从构造的角度解释 judgement 是如何产生的。可以看作是推导规则的逆过程。
		
		\begin{lemma} (子项引理)如果 M 是合法的,则 M 的每一个子项都是合法的。
		\end{lemma}
	
		\begin{lemma} (唯一性引理)假设$\Gamma\vdash M:\sigma$ 且 $\Gamma\vdash M:\tau$,则$\sigma\equiv\tau$。
		\end{lemma}
		
		唯一性引理也即在给定的上下文中,任意给定的项都只有唯一的类型。
		
	\section{规约和$\lambda{\rightarrow}$}
		\begin{lemma} (替换引理)假设$\Gamma',x:\sigma,\Gamma''\vdash M:\tau$且$\Gamma'\vdash N:\sigma$,则$\Gamma',\Gamma''\vdash M\left[x:=N\right]:\tau$。
		\end{lemma}
	
		再替换时,x 被替换为同类型的项,而 M 的类型不发生改变。

\end{document}
