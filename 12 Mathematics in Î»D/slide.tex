\documentclass[UTF8,aspectratio=169,mathserif]{beamer}
\usepackage{ctex}
\usepackage{ulem}
\usepackage{color}
\usepackage{amssymb}
\usepackage{amsmath}
\usetheme{Berlin}
\setbeamertemplate{navigation symbols}{}

\title{依赖于类型的类型}
\subtitle{Types Depending on Types}
\author{报告人:许博}
\date{}

\begin{document}
	
	\begin{frame}
		\titlepage
	\end{frame}

	\begin{frame}{目录}
		\tableofcontents
	\end{frame}

	\section{类型构造子}
		\begin{frame}{引入类型构造子}
			\begin{block}{$\lambda{\rightarrow}$}
				$\lambda x:\alpha.x$
			\end{block}
			
			\begin{block}{$\lambda{2},\ \lambda{\rightarrow}\ + $ terms depending on types}
				$\lambda\alpha:*.\lambda x:\alpha.x$
				
				$(\lambda\alpha:*.\lambda x:\alpha.x)\beta\rightarrow_\beta\lambda x:\beta.x$
			\end{block}
		
			\begin{block}{$\lambda{\underline{\omega}},\ \lambda{\rightarrow}\ + $ types depending on types}
				$\beta\rightarrow\beta,\gamma\rightarrow\gamma,...$ 等具有结构$\lozenge\rightarrow\lozenge$的类型,其中箭头的左右是一样的类型
				
				\textcolor{red}{$\lambda\alpha:*.\alpha\rightarrow\alpha$,类型构造子}
				
				$(\lambda\alpha:*.\alpha\rightarrow\alpha)\beta\rightarrow_\beta\beta\rightarrow\beta$
			\end{block}
		\end{frame}
	
		\begin{frame}[shrink]{类型构造子的类型,超级类型}
			\begin{block}{类型构造子的类型}
				$\alpha:*$
				
				$\alpha\rightarrow\alpha:*$
				
				$\lambda\alpha:*.\alpha\rightarrow\alpha:*\rightarrow*$
				
				$\lambda\alpha:*.\lambda\beta:*.\alpha\rightarrow\beta:*\rightarrow(*\rightarrow*)$
			\end{block}
			\begin{block}{超级类型,记为类(kind),$\mathbb{K}$}
				$*, *\rightarrow*, *\rightarrow(*\rightarrow*), ...$
				
				$\mathbb{K}=*|(\mathbb{K}\rightarrow\mathbb{K})$
			\end{block}
			\begin{block}{超级超级类型,$\square$,仅有一个}
				$*:\square, *\rightarrow*:\square, *\rightarrow(*\rightarrow*):\square, ...$
			\end{block}
		\end{frame}
	
		\begin{frame}[shrink]{类型构造子}
			\begin{exampleblock}{定义4.1.4(构造子,真构造子,类别)}
				(1) 如果$\kappa:\square$且$M:\kappa$,则$M$是一个构造子,如果$\kappa\not\equiv*$,则$M$是一个真构造子
				
				(2) 类别($s$)的集合为$\{*,\square\}$
			\end{exampleblock}
			\begin{columns}
				\begin{column}{0.48\textwidth}
					\begin{block}{类型构造子的定义}
						如果$\kappa$是一个类(kind),则对于每个类型是$\kappa$的M(即$M:\kappa$),M被称作是一个类型构造子,简称为构造子
						
						因为$*:\square,\alpha:*$,所以$\alpha$或者$\alpha\rightarrow\alpha$也都是构造子
					\end{block}
				\end{column}
				\begin{column}{0.475\textwidth}
					\begin{block}{真构造子}
						使用真构造子(proper constructor)表示不是类型的构造子(即类型不是$*$的构造子)。
					\end{block}
				\end{column}
			\end{columns}
			\begin{block}{类别(sort), s}
				使用类别(sort)或符号$s$表示$*$或$\square$。
			\end{block}
		\end{frame}
	
		\begin{frame}{四个层级(level)}
			\begin{exampleblock}{定义4.1.6(层级,level)}
				第1层:项(terms);
				
				第2层:构造子(包括类型和真构造子);
				
				第3层:类(kinds);
				
				第4层:超级超级类型$\square$。
			\end{exampleblock}
		
			对于语句$A:B$,可以得出$B$所处的层级一定比$A$高一级,比如当$A$是一个项时,$B$是一个类型,或者$A$是一个类型时,$B\equiv*$。
		\end{frame}

	\section{类别和变量规则}
		\begin{frame}{类别规则}
			\begin{exampleblock}{定义4.2.1(类别规则,sort-rule)}
				(sort) $\emptyset \vdash *:\square$
			\end{exampleblock}
		
			形式化$*$的类型是$\square$,$*:\square$可以由空的上下文推导出。
		\end{frame}
	
		\begin{frame}{良构(well-formed)的类型定义}
			给定的上下文中所有的声明都需要是可推导的。
			\begin{columns}
				\begin{column}{0.48\textwidth}
					\begin{block}{定义2.4.5($\lambda{\rightarrow}$的var-rule)}
						如果$x:\sigma\in\Gamma$,则$\Gamma\vdash x:\sigma$
					\end{block}
					对于$\lambda{\rightarrow}$上下文中出现的类型,合法类型的集合已经给出,可以直接判断合法与否。
				\end{column}
				\begin{column}{0.48\textwidth}
					\begin{block}{定义3.4.6($\lambda{2}$的var-rule)}
						如果$\Gamma$是一个$\lambda{2}$-上下文且$x:\sigma\in\Gamma$,则$\Gamma\vdash x:\sigma$
					\end{block}
					$\lambda{2}$的类型复杂一点,需要类型中的自由类型变量在上下文中已声明,因此类型的合法与否依赖于推定的上下文。
				\end{column}
			\end{columns}
		\end{frame}
		
		\begin{frame}{$\lambda{\uline{\omega}}$中的变量规则}
			\begin{exampleblock}{定义4.2.2(变量规则,var-rule)}
				(var) 如果$x\not\in\Gamma$,则$\dfrac{\Gamma\vdash A:s}{\Gamma,x:A\vdash x:A}$
			\end{exampleblock}
			推定中的类型的合法与否取决于是否可以被形式化地推导出。
			
			% 这里要求可形式化地推导是因为类型复杂且依赖上下文,不能通过预先生成的集合和上下文判断是否合法。类型跨越了两个层级,不能通过一个产生式生成集合。
			
			$x\not\in\Gamma$保证了变量$x$未在$\Gamma$中出现,因此声明在一个上下文中的所有的变量都是不同的,避免了变量名相同(类型不同)时造成的混淆。
		\end{frame}
	
		\begin{frame}
			\begin{block}{应用(sort)和(var)规则的一个例子}
				$\cfrac{\cfrac{(1)\ \emptyset\vdash*:\square}{(2)\ \alpha:*\vdash\alpha:*}\ (var)}{(3)\ \alpha:*,x:\alpha\vdash x:\alpha}\ (var)$
				
				只能推导出上下文中新添加的最后一个声明。
			\end{block}
		
			\begin{block}{不能推导的例子}
				$\alpha:*,x:\alpha\vdash\alpha:*$
				
				$\alpha:*,\beta:*\vdash\alpha:*$
				
				$\alpha:*,\beta:*\vdash\beta:*$
				
				最后一个例子中,不能得到$\bf premiss$:$\alpha:*\vdash*:\square$
			\end{block}
		\end{frame}

	\section{弱化规则(weakening rule)}
		\begin{frame}{$\lambda{\uline{\omega}}$中的弱化规则(weakening rule)}
			\begin{exampleblock}{定义4.3.1(弱化规则,weakening rule)}
				(weak) 如果$x\not\in\Gamma$,则$\dfrac{\Gamma\vdash A:B\ \ \ \Gamma\vdash C:s}{\Gamma,x:C\vdash A:B}$
			\end{exampleblock}
		
			若上下文$\Gamma$已经可以推导出$A:B$,在$\Gamma$的尾部添加了一个任意的声明(弱化)后仍然可以推导出$A:B$。需要注意的是添加的声明的类型需要是可推导出的。
		
			\begin{block}{推导$\alpha:*,x:\alpha\vdash\alpha:*$}
				$\cfrac{\cfrac{(1)\ \emptyset\vdash*:\square}{(2)\ \alpha:*\vdash\alpha:*}\ (var)\ \ \ \cfrac{(1)\ \emptyset\vdash*:\square}{(2)\ \alpha:*\vdash\alpha:*}\ (var)}{(3)\ \alpha:*,x:\alpha\vdash\alpha:*}\ (weak)$
			\end{block}
		
			% 在添加了这条规则之后,之前的稀疏引理便可以在$\lambda{\uline{\omega}}$中成立。
		\end{frame}

	\section{形成规则(formation rule)}
		\begin{frame}
			\begin{columns}
				\begin{column}{0.48\textwidth}
					\begin{exampleblock}{定义4.4.1(形成规则,formation rule)}
						(form)\ $\dfrac{\Gamma\vdash A:s\ \ \ \Gamma\vdash B:s}{\Gamma\vdash A\rightarrow B:s}$
					\end{exampleblock}
				\end{column}
				\begin{column}{0.48\textwidth}
					\begin{block}{定义3.4.7($\lambda{\rightarrow}$中的形成规则,formation rule in $\lambda{\rightarrow}$)}
						(form) 如果$\Gamma$是一个$\lambda{2}$-上下文,$B\in\mathbb{T}2$且$B$中所有的自由类型变量已经在$\Gamma$中声明,则$\Gamma\vdash B:*$
					\end{block}
				\end{column}
			\end{columns}
			
			三个$s$同时表示*或$\square$。
			
			在引入新的(form)规则之前,由(var)规则与(weak)规则只能推导出$*:\square,\alpha:*,x:\alpha$而不能推出箭头类型$*\rightarrow*:\square$以及$\alpha\rightarrow\beta:*$等。
			
			\begin{block}{例子}
				$\dfrac{\emptyset\vdash*:\square\ \ \ \emptyset\vdash*:\square}{\emptyset\vdash*\rightarrow*:\square}\ (form)$
			\end{block}
			
		\end{frame}

	\section{应用与抽象规则}
		\begin{frame}
			\begin{columns}
				\begin{column}{0.48\textwidth}
					\begin{exampleblock}{应用规则,application rule}
						(appl) $\dfrac{\Gamma\vdash M:A\rightarrow B\ \ \ \Gamma\vdash N:A}{\Gamma\vdash MN:B}$
					\end{exampleblock}
				\end{column}
				\begin{column}{0.48\textwidth}
					\begin{exampleblock}{抽象规则,abstraction rule}
						(abst) $\dfrac{\Gamma,x:A\vdash M:B\ \ \ \Gamma\vdash A\rightarrow B:s}{\Gamma\vdash\lambda x:A.M:A\rightarrow B}$
					\end{exampleblock}
				\end{column}
			\end{columns}
		
			\hspace*{\fill} \\
			
			$\lambda{\uline{\omega}}$中的这两个规则与之前略有不同:
			
			\begin{itemize}
				\item 用于类型的元变量(meta-variable)的名字不同,因为$\lambda{\uline{\omega}}$中的类型更为通用
				
				\item 需要保证类型是良构的(即可以由上下文推导出)
			\end{itemize}
			
			当$s\equiv*$时,$A\rightarrow B$是一个第二层的类型,如$(\alpha\rightarrow\beta)\rightarrow\gamma$,
			
			当$s\equiv\square$时,$A\rightarrow B$就是一个第三层的类(kind),如$(*\rightarrow*)\rightarrow*$。
		\end{frame}

	\section{简化(shortened)推导}
		\begin{frame}{简化(shortened)推导}
			\begin{block}{推导$\alpha:*,\beta:*\vdash\alpha\rightarrow\beta:*$,需要以下推定:}
				$\emptyset\vdash*:\square$ (sort),
				
				$\alpha:*\vdash\alpha:*$ (var),
				
				$\alpha:*\vdash*:\square$ (weak),
				
				$\alpha:*,\beta:*\vdash\alpha:*$ (weak),
				
				$\alpha:*,\beta:*\vdash\beta:*$ (var)。
			\end{block}
			
			允许跳过以下推导过程:
			
			(i) 当使用规则(sort),(var)和(weak)时,
			
			(ii) 当使用规则(form)时,以及
			
			(iii) 当确定(abst)规则的第二个$\bf premiss$的合法性时。
			
			因此$\alpha:*,\beta:*\vdash\alpha\rightarrow\beta:*$现在可以直接使用。
		\end{frame}

	\section{变换(conversion)规则}
		\begin{frame}
			\begin{exampleblock}{定义4.7.1(变换规则,conversion rule)}
				(conv) 如果$B=_\beta B'$,则$\dfrac{\Gamma\vdash A:B\ \ \ \Gamma\vdash B':s}{\Gamma\vdash A:B'}$
			\end{exampleblock}
		
			$B$作为已知推定$\Gamma\vdash A:B$的部分是良构的,$\Gamma\vdash B':s$保证$B'$也是良构的。而$B=_\beta B'$无法保证$B'$是良构的,比如$\beta\rightarrow\gamma=_\beta(\lambda\alpha:*.\beta\rightarrow\gamma)M$。
			
			\begin{block}{例子}
				令$\Gamma\equiv\beta:*,x:(\lambda\alpha:*.\alpha\rightarrow\alpha)\beta$:
				
				$\cfrac{\Gamma\vdash x:(\lambda\alpha:*.\alpha\rightarrow\alpha)\beta\ \ \ \Gamma\vdash\beta\rightarrow\beta:*}{\Gamma\vdash x:\beta\rightarrow\beta}$ (conv)
			\end{block}
		\end{frame}
	
		\begin{frame}[shrink]{$\lambda{\uline{\omega}}$-规则}
			(sort) $\emptyset \vdash *:\square$
			
			(var) 如果$x\not\in\Gamma$,则$\cfrac{\Gamma\vdash A:s}{\Gamma,x:A\vdash x:A}$
			
			(weak) 如果$x\not\in\Gamma$,则$\cfrac{\Gamma\vdash A:B\ \ \ \Gamma\vdash C:s}{\Gamma,x:C\vdash A:B}$
			
			(form)\ $\cfrac{\Gamma\vdash A:s\ \ \ \Gamma\vdash B:s}{\Gamma\vdash A\rightarrow B:s}$
			
			(appl) $\cfrac{\Gamma\vdash M:A\rightarrow B\ \ \ \Gamma\vdash N:A}{\Gamma\vdash MN:B}$
			
			(abst) $\cfrac{\Gamma,x:A\vdash M:B\ \ \ \Gamma\vdash A\rightarrow B:s}{\Gamma\vdash\lambda x:A.M:A\rightarrow B}$
			
			(conv) 如果$B=_\beta B'$,则$\cfrac{\Gamma\vdash A:B\ \ \ \Gamma\vdash B':s}{\Gamma\vdash A:B'}$\\
		\end{frame}

	\section{性质}
		\begin{frame}{$\lambda{\uline{\omega}}$的性质}
			$\lambda{\uline{\omega}}$满足之前几章所提到的大部分性质,但是类型唯一性引理需要进行调整:
			
			\begin{exampleblock}{引理4.8.1(类型唯一性引理)}
				如果$\Gamma\vdash A:B_1$且$\Gamma\vdash A:B_2$,则$B_1=_\beta B_2$
			\end{exampleblock}
		
			\begin{block}{引理2.10.9(适用于$\lambda{\rightarrow}$的类型唯一性引理)}
				假设$\Gamma\vdash M:\sigma$且$\Gamma\vdash M:\tau$,则$\sigma\equiv\tau$
			\end{block}
		\end{frame}
	
\end{document}